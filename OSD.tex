\documentclass{article}

\title{
	Optimal Subgroup Discovery
}
\author{
	Jakob Bach~\orcidlink{0000-0003-0301-2798}\\
	\small Karlsruhe Institute of Technology (KIT), Germany\\
	\small \href{mailto:jakob.bach@kit.edu}{jakob.bach@kit.edu}
}
\date{} % don't display a date

\usepackage[style=numeric, backend=bibtex]{biblatex}
\usepackage[ruled,linesnumbered,vlined]{algorithm2e} % pseudo-code
\usepackage{amsmath} % mathematical symbols
\usepackage{amssymb} % mathematical symbols
\usepackage{amsthm} % theorems, definitions etc.
\usepackage{booktabs} % nicely formatted tables (with top, mid, and bottom rule)
\usepackage{graphicx} % plots
\usepackage{orcidlink} % ORCID icon
\usepackage{subcaption} % figures with multiple sub-figures and sub-captions
\usepackage{hyperref} % links and URLs

\addbibresource{references.bib}

\newtheorem{proposition}{Proposition}
\theoremstyle{definition}
\newtheorem{definition}{Definition}

\begin{document}

\maketitle

\begin{abstract}
\end{abstract}
%
\textbf{Keywords:} subgroup discovery, alternatives, constraints, satisfiability modulo theories, mixed-integer programming, explainability, interpretability, XAI

\section{Introduction}
\label{sec:osd:introduction}

\paragraph{Motivation}

\cite{carvalho2019machine} \cite{molnar2020interpretable}

\paragraph{Problem statement}

\paragraph{Related work}

\paragraph{Contributions}

\paragraph{Experimental results}

\paragraph{Outline}

Section~\ref{sec:osd:fundamentals} introduces notation and fundamentals.
Section~\ref{sec:osd:approach} describes and analyzes alternative feature selection.
Section~\ref{sec:osd:related-work} reviews related work.
Section~\ref{sec:osd:experimental-design} outlines our experimental design, while Section~\ref{sec:osd:evaluation} presents the experimental results.
Section~\ref{sec:osd:conclusion} concludes.
Appendix~\ref{sec:osd:appendix} contains supplementary materials.

\section{Fundamentals}
\label{sec:osd:fundamentals}

\subsection{Notation}
\label{sec:osd:fundamentals:notation}

$X \in \mathbb{R}^{m \times n}$ stands for a dataset in the form of a matrix.
Each row is a data object, and each column is a feature.
$F = \{f_1, \dots, f_n\}$ is the corresponding set of feature names.
We assume that categorical features have already been made numeric, e.g., via one-hot encoding.
$X_{\cdot{}j} \in \mathbb{R}^m$ denotes the vector representation of the $j$-th feature.
$y \in Y^m$ represents the prediction target with domain $Y$, e.g., $Y=\{0,1\}$ for binary classification or $Y=\mathbb{R}$ for regression.

In subgroup discovery, one ... (goal, decision variables)
The function $Q(...,X,y)$ returns the quality of such a subgroup.
Without loss of generality, we assume that this function should be maximized.

\section{Approach}
\label{sec:osd:approach}

\subsection{Optimization Problem}
\label{sec:osd:approach:problem}

MaxSAT: \cite{li2021maxsat} \cite{bacchus2021maximum}
SMT: \cite{barrett2018satisfiability}

\subsection{Constraints}
\label{sec:osd:approach:constraints}

\cite{mosek2022modeling}
\cite{sinz2005towards}
\cite{ulrich2022selecting}

\subsection{Objective Functions}
\label{sec:osd:approach:objectives}

\subsection{Computational Complexity}
\label{sec:osd:approach:complexity}

\section{Related Work}
\label{sec:osd:related-work}

\subsection{Subgroup Discovery}
\label{sec:osd:related-work:subgroup-discovery}

\cite{leeuwen2012diverse}
\cite{arzamasov2021reds} \cite{arzamasov2022pedagogical} \cite{vollmer2019informative}

\subsection{Feature Selection}
\label{sec:osd:related-work:feature-selection}

\cite{bach2022empirical} \cite{bach2023finding}

\subsection{Other Fields}
\label{sec:osd:related-work:other}

\cite{bailey2014alternative} \cite{grossi2017survey}
\cite{guidotti2022counterfactual}
\cite{narodytska2018learning} \cite{schidler2021sat} \cite{yu2021learning}

\section{Experimental Design}
\label{sec:osd:experimental-design}

\subsection{Overview}
\label{sec:osd:experimental-design:overview}

\subsection{Evaluation Metrics}
\label{sec:osd:experimental-design:evaluation}

\paragraph{Quality}

\paragraph{Runtime}

\subsection{Methods}
\label{sec:osd:experimental-design:methods}

\subsection{Datasets}
\label{sec:osd:experimental-design:datasets}

\subsection{Implementation and Execution}
\label{sec:osd:experimental-design:implementation}

\cite{bestuzheva2021scip}
\cite{deMoura2008z3}
\cite{perron2022or-tools}

\section{Evaluation}
\label{sec:osd:evaluation}

\subsection{Summary}
\label{sec:osd:evaluation:summary}

\section{Conclusions and Future Work}
\label{sec:osd:conclusion}

\subsection{Conclusions}
\label{sec:osd:conclusion:conclusion}

\subsection{Future Work}
\label{sec:osd:conclusion:future-work}

%~\\
%\noindent \textsc{Acknowledgments}\quad
%This work was supported by the Ministry of Science, Research and the Arts Baden-Württemberg, project \emph{Algorithm Engineering for the Scalability Challenge (AESC)}.

\appendix

\section{Appendix}
\label{sec:osd:appendix}

In this section, we provide supplementary materials.

\renewcommand*{\bibfont}{\small} % use a smaller font for bib than for main text
\printbibliography

\end{document}
